
% License:
% CC BY-NC-SA 3.0 (http://creativecommons.org/licenses/by-nc-sa/3.0/)
%
%%%%%%%%%%%%%%%%%%%%%%%%%%%%%%%%%%%%%%%%%

%----------------------------------------------------------------------------------------
%	PACKAGES AND OTHER DOCUMENT CONFIGURATIONS
%----------------------------------------------------------------------------------------

\documentclass[paper=a4, fontsize=16pt]{scrartcl} % A4 paper and 11pt font size

\usepackage[T1]{fontenc} % Use 8-bit encoding that has 256 glyphs
\usepackage{fourier} % Use the Adobe Utopia font for the document - comment this line to return to the LaTeX default
\usepackage[english]{babel} % English language/hyphenation
\usepackage{amsmath,amsfonts,amsthm} % Math packages
\usepackage{lipsum} % Used for inserting dummy 'Lorem ipsum' text into the template

\usepackage{caption}
\usepackage{subcaption}
\usepackage{graphicx}

\usepackage{float}

\usepackage{blindtext} %for enumarations

\usepackage[]{hyperref}  %link collor

%talbe layout to the right
%\usepackage[labelfont=bf]{caption}
%\captionsetup[table]{labelsep=space,justification=raggedright,singlelinecheck=off}
%\captionsetup[figure]{labelsep=quad}

\usepackage{sectsty} % Allows customizing section commands
\allsectionsfont{\centering \normalfont\scshape} % Make all sections centered, the default font and small caps

\usepackage{fancyhdr} % Custom headers and footers
\pagestyle{fancyplain} % Makes all pages in the document conform to the custom headers and footers
\fancyhead{} % No page header - if you want one, create it in the same way as the footers below
\fancyfoot[L]{} % Empty left footer
\fancyfoot[C]{} % Empty center footer
\fancyfoot[R]{\thepage} % Page numbering for right footer
\renewcommand{\headrulewidth}{0pt} % Remove header underlines
\renewcommand{\footrulewidth}{0pt} % Remove footer underlines
\setlength{\headheight}{13.6pt} % Customize the height of the header

\numberwithin{equation}{section} % Number equations within sections (i.e. 1.1, 1.2, 2.1, 2.2 instead of 1, 2, 3, 4)
\numberwithin{figure}{section} % Number figures within sections (i.e. 1.1, 1.2, 2.1, 2.2 instead of 1, 2, 3, 4)
\numberwithin{table}{section} % Number tables within sections (i.e. 1.1, 1.2, 2.1, 2.2 instead of 1, 2, 3, 4)

%\setlength\parindent{0pt} % Removes all indentation from paragraphs - comment this line for an assignment with lots of text


\setlength\parskip{4pt}

%----------------------------------------------------------------------------------------
%	TITLE SECTION
%----------------------------------------------------------------------------------------

\newcommand{\horrule}[1]{\rule{\linewidth}{#1}} % Create horizontal rule command with 1 argument of height

\title{	
\normalfont \normalsize 
\textsc{Indian Institute of Technology, Bombay} \\ [50pt] % Your university, school and/or department name(s)
\horrule{0.5pt} \\[0.4cm] % Thin top horizontal rule
\huge CS699 Project : Mess Management system\\ % The assignment title
\horrule{2pt} \\[0.5cm] % Thick bottom horizontal rule
}

\author{  Yadnyesh Patil, 193050067\\
Aman Kumar Singh, 193050022 \\ Ashish Kumar Goyal, 193050058} % Your name

\date{\normalsize\today} % Today's date or a custom date
\begin{document}
%\nocite{*}
 % Print the title
\clearpage\maketitle
\thispagestyle{empty}
\newpage
\begin{centering}


\begin{abstract}
\thispagestyle{empty}
{\hspace{5.5cm}\Large ABSTRACT}\\
\bigbreak
In this project our main aim is to develop a mess management system which eases the process of making and approving rebate or overhead request. It also helps the mess staff to create bills at the end of the month.   \\

  
  
\bigbreak

\textbf{Key words:} mess management, rebate and overhead request, monthlty bills.
    
\end{abstract}

\end{centering}
\vspace{\fill}

\newpage
\tableofcontents


%----------------------------------------------------------------------------------------
%	Section 1
%----------------------------------------------------------------------------------------

%how to cite
%\cite{Seow2011}
%how to add figure
\newpage
\section{Introduction}

In this project, we implemented a webapp using python with django framework. This webapp facilitates the process of applying for rebates and overhead. It provides an interface for students to apply for rebates and overhead. The mess worker or mess manager will be given a separate role in the app through which they can approve those requests. 

There will be three type of users:
\begin{flushleft}
\begin{enumerate}
    \item Student : A Student can request for rebate and overhead through forms available in the pages accessible to him/her. He/She can also see the history of requests by him/her.
    \item Mess Worker : Mess worker here have access to his separate pages using which he can approve overhead requests as these requests are usually dealt by mess worker himself. We also provide the functionality of changing the prices of the meal.
    \item Mess Manager : Mess manager approves the rebate requests as done so in the current scenario. Rebate requests needs to be analyzed before approving that's why it requires mess manager to do that.
\end{enumerate}
\end{flushleft}
\begin{flushleft}
At the end of every month bills need to generated for which a lot of paperwork is needed. Since the app keeps track of overheads and rebates of every student, it makes the process of generating the monthly bills really easy. 
\end{flushleft}








\newpage
\section{Motivation}
The following reasons were mostly important for us choosing and working on this project:
\begin{flushleft}
\begin{itemize}
    \item The current process of applying for rebates is not very clear and hence most of the students just don't apply for it which in turn wastes food,money etc.
    \item If the mess workers know the number of students that are going to miss the meal, it will be easy for them to estimate the amount of food to be made.
    \item The record of students are stored on a register, one page for each student and most of the time these pages go blank at the end of the month. This incurs a lot of wastage of paper which can be easily avoided.
    \item The effort for creating the mess bills of all the students at the end of the month is a tedious task which can very well be made easy using the app.Less efforts and better results for every one
\end{itemize}
\end{flushleft}



\newpage
\section{Database Tables}
\begin{itemize}
    \item Student: To store the information about the student. It  will contain the attributes like: \\
    Roll No.\\
    Mess Card No.\\
    Room No.\\
    Name\\
    \item Rebate request: To store the rebate requests made by the students in the rebate re table. Attributes will be like:\\Roll No.\\From(date)\\To(date)\\
    
    \item overhead request: To store the info of all the overhead request made by students Attributes will be like:\\Roll No.\\meal\\No. of overheads\\
    
    \item Meals: To store the information of the meals to be catered at different times of the day. Attributes will be like:\\Meal type\\ price \\type(veg, non-veg)\\
    
    
    
\end{itemize}

\newpage
\section{User Manual}
User Manual for different users:
\begin{enumerate}
    \item Student: The following guidelines can help students to use this app:\\
    \begin{itemize}
        \item Contact mess manager so that he can add you the app. Since this is done once a year under manager's supervision. So no requirement of sign up page
        \item Go to the links for requests provided in the homepage of student 'user' and apply for rebate or overhead by filling the form.
        \item Can see previous requests made by him/her on the homepage of student type user.
    \end{itemize}
    \item Mess Worker: These guidelines as useful for mess worker:\\
    \begin{itemize}
        \item Can see overhead requests of the students and approve them on his homepage.
        \item Can access the meals page from link provided on his homepage to change prices of the meals.
        \end{itemize}
    \item Mess Manager:The following guidelines can help Mess manager to use this app: \\
    \begin{itemize}
        \item Listen to students and add or delete them from the database
        \item Can see the rebate requests on his homepage and  approve or reject them according to his convenience.
        \item can add mess worker into the database using the admin console provided by django framework.
    \end{itemize}
    
\end{enumerate}




\newpage
\section{Future Scope}
\begin{itemize}
    \item  As we can see that the app really is very practical and can be readily implemented in a Standard mess.
    \item There still are a lot of functionalities that can be implemented here like feature of special meal, dividing students on the basis of type of food they eat etc.
    \item The app can also be implemented on a large scale by adding a little more features and by making it a bit more flexible.
    
\end{itemize}


\end{document}
