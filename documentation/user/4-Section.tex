\newpage
\section{Database Tables}
\begin{itemize}
    \item Student: To store the information about the student. It  will contain the attributes like: \\
    Roll No.\\
    Mess Card No.\\
    Room No.\\
    Name\\
    \item Rebate request: To store the rebate requests made by the students in the rebate re table. Attributes will be like:\\Roll No.\\From(date)\\To(date)\\
    
    \item overhead request: To store the info of all the overhead request made by students Attributes will be like:\\Roll No.\\meal\\No. of overheads\\
    
    \item Meals: To store the information of the meals to be catered at different times of the day. Attributes will be like:\\Meal type\\ price \\type(veg, non-veg)\\
    
    
    
\end{itemize}

\newpage
\section{User Manual}
User Manual for different users:
\begin{enumerate}
    \item Student: The following guidelines can help students to use this app:\\
    \begin{itemize}
        \item Contact mess manager so that he can add you the app. Since this is done once a year under manager's supervision. So no requirement of sign up page
        \item Go to the links for requests provided in the homepage of student 'user' and apply for rebate or overhead by filling the form.
        \item Can see previous requests made by him/her on the homepage of student type user.
    \end{itemize}
    \item Mess Worker: These guidelines as useful for mess worker:\\
    \begin{itemize}
        \item Can see overhead requests of the students and approve them on his homepage.
        \item Can access the meals page from link provided on his homepage to change prices of the meals.
        \end{itemize}
    \item Mess Manager:The following guidelines can help Mess manager to use this app: \\
    \begin{itemize}
        \item Listen to students and add or delete them from the database
        \item Can see the rebate requests on his homepage and  approve or reject them according to his convenience.
        \item can add mess worker into the database using the admin console provided by django framework.
    \end{itemize}
    
\end{enumerate}

